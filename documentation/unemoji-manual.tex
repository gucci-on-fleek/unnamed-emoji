% unnamed-emoji
% https://github.com/gucci-on-fleek/unnamed-emoji
% SPDX-License-Identifier: MPL-2.0+ OR CC-BY-SA-4.0+
% SPDX-FileCopyrightText: 2023 Max Chernoff

\doifnot{\contextmark}{LMTX}{
    \errhelp{LMTX/MkXL is required to compile this file.}
    \errmessage{Fatal error, exiting.}
}

% From <texmf-dist/source/luatex/lua-widow-control/lwc-manual.mkxl>
% or <https://raw.githubusercontent.com/gucci-on-fleek/lua-widow-control/master/docs/manual/lwc-manual.mkxl>
\environment lwc-manual

% Abbreviations
\def\unemoji/{\sans{unnamed-emoji}}
\def\Unemoji/{\sans{Unnamed-emoji}}
\useURL[projecturl][https://github.com/gucci-on-fleek/unnamed-emoji]
\let\q=\quotation

% Define the example snippets
\usemodule[scite]

\startluacode

    -- Always use document class standalone for the demos
    interfaces.implement {
        name = "fixbuffers",
        public = true,
        actions = function()
            local buff = buffers.raw("latex")
            buffers.assign("code", buff)
            buffers.assign("latex", buff:gsub("\\documentclass{.-}", "\\documentclass[border=1pt]{standalone}"))
        end,
    }

    -- Let us use snippets instead of complete documents
    interfaces.implement {
        name = "buffersnippet",
        public = true,
        actions = function()
            local buff = buffers.raw("latex")
            buffers.assign("code", buff)
            buffers.assign(
                "latex",
                "\\documentclass[border=1pt]{standalone}" ..
                "\\usepackage{unnamed-emoji}" ..
                "\\begin{document}" ..
                buff ..
                "\\end{document}"
            )
        end,
    }

    sandbox.registerrunner {
        name     = "latex",
        program  = "pdflatex",
        method   = "execute",
        template = "%filename%",
        checkers = {
            filename = "readable",
            path     = "string",
        }
    }
\stopluacode

% Define the frame-drawing command
\startMPcode
    def drawframe (expr p) =
        draw p withpen pencircle scaled 1mm dashed dashpattern(on 0pt off 4mm) withcolor 0.7white;
        draw p withpen pencircle scaled 1mm dashed dashpattern(off 2mm on 0pt off 2mm) withcolor 0.3white;
        setbounds currentpicture to boundingbox OverlayBox;
    enddef;
\stopMPcode

% Full rounded frame
\startuniqueMPgraphic{outputframe}
    path p;
    p := roundedsquare(OverlayWidth, OverlayHeight, 0.25cm);
    drawframe(p);
\stopuniqueMPgraphic
\defineoverlay[outputframe][\useMPgraphic{outputframe}]

% Middle divider line
\startuniqueMPgraphic{divider}
    path p;
    p := (OverlayWidth - 0.25cm, -0.17cm)
         -- (OverlayWidth - 0.25cm, OverlayHeight + 0.17cm);
    drawframe(p);
\stopuniqueMPgraphic
\defineoverlay[divider][\useMPgraphic{divider}]

\unprotect
% Define the full drawing command
\define[1]\outputframe{
    \startframedtext[
        background=outputframe,
        width=fit,
        align=flushleft,
        frame=off,
        height=fit,
        offset=1ex,
    ]
        \setupTABLE[c][width=0.45\textwidth]
        \setupTABLE[c][1][background=divider]
        \startTABLE[offset=0pt, frame=off]
        \NC \setupbodyfont[9.5pt]
            \typeTEXbuffer[code]

        \NC \externalfigure[#1][
            maxwidth=\dimexpr 0.4\textwidth,
            maxheight=0.9\textheight,
        ]

        \NC\NR
        \stopTABLE
    \stopframedtext
}

% Full document commands
\define\startlatex{%
    \grabbufferdata[latex][startlatex][stoplatex]%
}

\def\stoplatex{%
    \fixbuffers%
    \runbuffer[latex][latex]%
    \outputframe{\lasttypesetbuffer}%
    \spac_indentation_variant_no%
}

% Snippet commands
\define\startlatexsnippet{%
    \grabbufferdata[latex][startlatexsnippet][stoplatexsnippet]%
}

\def\stoplatexsnippet{%
    \buffersnippet%
    \runbuffer[latex][latex]%
    \outputframe{\lasttypesetbuffer}%
    \spac_indentation_variant_no%
}%
\protect

\startdocument[
    title=unnamed-emoji,
    author=Max Chernoff,
    version=0.0.3, %%version
    github=https://github.com/gucci-on-fleek/unnamed-emoji,
]

\Unemoji/ is an experimental emoji package for \LaTeX{} and Plain~\TeX{}. It
natively supports \pdfTeX{} and \LuaTeX{}, and can support \XeTeX{} and
\type{dvipdfmx} with a patched version of \type{dvipdfmx}.

\subject{Contents}
\placecontent[criterium=all]

\page
\setuplayout[
    width=middle,
    backspace=1in,
    height=9.25in,
]
\section[sec:implementation]{Implementation}

From here and until the end of this manual is the raw source code of \unemoji/. This is primarily of interest to developers; most users need not read further.

If want to offer any improvements to the code below, please open an issue or a \acronym{PR} on \goto{GitHub}[url(projecturl)].

\setupbodyfont[10pt]
\setuphead[subsection][
    alternative=normal,
    style=\ssitb,
    after={\blank[disable, penalty:10000]},
    page=yes,
    continue=yes,
]

\subsection{svg-to-pdf.cld}
\typeLUAfile{../source/svg-to-pdf.cld}

\subsection{unnamed-emoji.sty}
\typeTEXfile{../source/unnamed-emoji.sty}

\subsection{unnamed-emoji.tex}
\typeTEXfile{../source/unnamed-emoji.tex}

\subsection{unnamed-emoji.lua}
\typeLUAfile{../source/unnamed-emoji.lua}

\subsection{dvipdfmx.patch}
\typefile{../dvipdfmx.patch}

\stopdocument
