% unnamed-emoji
% https://github.com/gucci-on-fleek/unnamed-emoji
% SPDX-License-Identifier: MPL-2.0+ OR CC-BY-SA-4.0+
% SPDX-FileCopyrightText: 2023 Max Chernoff

\doifnot{\contextmark}{LMTX}{
    \errhelp{LMTX/MkXL is required to compile this file.}
    \errmessage{Fatal error, exiting.}
}

% From <texmf-dist/source/luatex/lua-widow-control/lwc-manual.mkxl>
% or <https://raw.githubusercontent.com/gucci-on-fleek/lua-widow-control/master/docs/manual/lwc-manual.mkxl>
\environment lwc-manual

% Abbreviations
\def\unemoji/{\sans{unnamed-emoji}}
\def\Unemoji/{\sans{Unnamed-emoji}}
\useURL[projecturl][https://github.com/gucci-on-fleek/unnamed-emoji]
\let\q=\quotation

% Define the example snippets
\usemodule[scite]

\startluacode

    -- Always use document class standalone for the demos
    interfaces.implement {
        name = "fixbuffers",
        public = true,
        actions = function()
            local buff = buffers.raw("latex")
            local preamble = buffers.raw("preamble")
            buffers.assign("preamble", "")
            buffers.assign("code", buff)
            buffers.assign("latex", preamble .. buff:gsub("\\documentclass{.-}", "\\documentclass[border=1pt]{standalone}"))
        end,
    }

    -- Let us use snippets instead of complete documents
    interfaces.implement {
        name = "buffersnippet",
        public = true,
        actions = function()
            local buff = buffers.raw("latex")
            buffers.assign("code", buff)
            buffers.assign(
                "latex",
                "\\documentclass[border=1pt]{standalone}" ..
                "\\usepackage{unnamed-emoji}" ..
                "\\begin{document}" ..
                buff ..
                "\n\\end{document}"
            )
        end,
    }

    sandbox.registerrunner {
        name     = "latex",
        program  = "lualatex",
        method   = "execute",
        template = "%filename%",
        checkers = {
            filename = "readable",
            path     = "string",
        }
    }

    sandbox.registerrunner {
        name     = "plain",
        program  = "pdftex",
        method   = "execute",
        template = "%filename%",
        checkers = {
            filename = "readable",
            path     = "string",
        }
    }
\stopluacode

% Define the frame-drawing command
\startMPcode
    def drawframe (expr p) =
        draw p withpen pencircle scaled 1mm dashed dashpattern(on 0pt off 4mm) withcolor 0.7white;
        draw p withpen pencircle scaled 1mm dashed dashpattern(off 2mm on 0pt off 2mm) withcolor 0.3white;
        setbounds currentpicture to boundingbox OverlayBox;
    enddef;
\stopMPcode

% Full rounded frame
\startuniqueMPgraphic{outputframe}
    path p;
    p := roundedsquare(OverlayWidth, OverlayHeight, 0.25cm);
    drawframe(p);
\stopuniqueMPgraphic
\defineoverlay[outputframe][\useMPgraphic{outputframe}]

% Middle divider line
\startuniqueMPgraphic{divider}
    path p;
    p := (OverlayWidth - 0.25cm, -0.17cm)
         -- (OverlayWidth - 0.25cm, OverlayHeight + 0.17cm);
    drawframe(p);
\stopuniqueMPgraphic
\defineoverlay[divider][\useMPgraphic{divider}]

\unprotect
% Define the full drawing command
\define[2]\outputframe{
    \startframedtext[
        background=outputframe,
        width=fit,
        align=flushleft,
        frame=off,
        height=fit,
        offset=1ex,
    ]
        \setupTABLE[c][1][width=0.65\textwidth, background=divider]
        \setupTABLE[c][2][width=0.30\textwidth]
        \startTABLE[offset=0pt, frame=off, align=top]
        \NC \setupbodyfont[9.5pt]
            \typeTEXbuffer[code]

        \NC \externalfigure[#1][
            factor=fit,
            maxwidth=\dimexpr 0.29\textwidth,
            #2
        ]

        \NC\NR
        \stopTABLE
    \stopframedtext
}

% Full document commands
\def\extrafigureparams{maxheight=2\baselineskip,width=0.29\textwidth,}

\define\startlatex{%
    \grabbufferdata[latex][startlatex][stoplatex]%
}

\def\stoplatex{%
    \fixbuffers%
    \runbuffer[latex][latex]%
    \normalexpanded{\noexpand\outputframe{\lasttypesetbuffer}{\extrafigureparams}}%
    \spac_indentation_variant_no%
}

% Snippet commands
\define\startlatexsnippet{%
    \grabbufferdata[latex][startlatexsnippet][stoplatexsnippet]%
}

\def\stoplatexsnippet{%
    \buffersnippet%
    \runbuffer[latex][latex]%
    \normalexpanded{\noexpand\outputframe{\lasttypesetbuffer}{\extrafigureparams}}%
    \spac_indentation_variant_no%
}%

% Plain
\define\startplain{%
    \grabbufferdata[latex][startplain][stopplain]%
}

\def\stopplain{%
    \fixbuffers%
    \runbuffer[latex][plain]%
    \normalexpanded{\noexpand\outputframe{\lasttypesetbuffer}{\extrafigureparams}}%
    \spac_indentation_variant_no%
}
\protect

\startdocument[
    title=unnamed-emoji,
    author=Max Chernoff,
    version=0.0.3, %%version
    github=https://github.com/gucci-on-fleek/unnamed-emoji,
]

\Unemoji/ is an experimental emoji package for \LaTeX{} and Plain~\TeX{}. It
natively supports \pdfTeX{} and \LuaTeX{}, and can support \XeTeX{} and
\type{dvipdfmx} with a patched version of \type{dvipdfmx}.

\subject{Contents}
\placecontent[criterium=all]

\section{Loading}

\subsection{\LaTeX{}}
\startlatex
\documentclass{article}

\usepackage{unnamed-emoji}

\begin{document}
    \emoji{duck}
\end{document}
\stoplatex

\subsection{Plain \TeX{}}
\startbuffer[preamble]
\pdfhorigin=0in
\pdfvorigin=0in
\pdfpagewidth=1em
\pdfpageheight=\baselineskip
\parindent=0pt
\stopbuffer
\startplain
\input unnamed-emoji

\emoji{duck}
\bye
\stopplain

\section{Input Formats}

\subsection{Name}
\startlatexsnippet
\emoji{duck}
\stoplatexsnippet

\subsection{Hexadecimal}
\startlatexsnippet
\emoji{1f986}
\stoplatexsnippet

\subsection{Decimal}
\startlatexsnippet
\emoji{129414}
\stoplatexsnippet

\subsection{Unicode}
\startlatexsnippet
\emoji{🦆}  % LuaTeX Only!
\stoplatexsnippet

\section{Font Selection}
\subsection{\LaTeX{}}
\begingroup
\def\extrafigureparams{width=0.29\textwidth}
\startlatexsnippet
\begin{tabular}{r@{: }l}
    \texttt{noto-emoji} &
    \emoji[noto-emoji]{1f638} \\
    %
    \texttt{twemoji} &
    \emoji[twemoji]{1f638} \\
    %
    \texttt{fxemoji} &
    \emoji[fxemoji]{1f638} \\
    %
    \texttt{openmoji} &
    \emoji[openmoji]{1f638} \\
    %
    \texttt{emojione} &
    \emoji[emojione]{1f638} \\
\end{tabular}
\stoplatexsnippet
\endgroup

\subsection{Plain \TeX{}}
\begingroup
\def\extrafigureparams{width=0.29\textwidth}
\startbuffer[preamble]
\pdfhorigin=0in
\pdfvorigin=0in
\pdfpagewidth=1.2in
\pdfpageheight=6\baselineskip
\parindent=0pt
\stopbuffer
\startplain
\input unnamed-emoji
\halign{\hfil\tt#\unskip:&#\hfil\cr
    noto-emoji &
    \def\emojifont{noto-emoji}
    \emoji{1f638} \cr
    %
    twemoji &
    \def\emojifont{twemoji}
    \emoji{1f638} \cr
    %
    fxemoji &
    \def\emojifont{fxemoji}
    \emoji{1f638} \cr
    %
    openmoji &
    \def\emojifont{openmoji}
    \emoji{1f638} \cr
    %
    emojione &
    \def\emojifont{emojione}
    \emoji{1f638} \cr
}
\bye
\stopplain
\endgroup

\section{Composed Emoji}

\subsection{Name}
\startlatexsnippet
\emoji[twemoji]
  {regional indicator symbol letter c
   regional indicator symbol letter a}
\stoplatexsnippet

\subsection{Hexadecimal}
\startlatexsnippet
\emoji[twemoji]{1f1e8-1f1e6}
\stoplatexsnippet

\section{Licence}

Most files should list their licence near the top. In general, the code
is licensed under the \goto{{\em Mozilla Public License}, version
2.0}[url(https://www.mozilla.org/en-US/MPL/2.0/)] or greater. The
documentation is additionally licensed under \goto{CC-BY-SA, version
4.0}[url(https://creativecommons.org/licenses/by-sa/4.0/legalcode)] or
greater.

\subsection{\type{noto-emoji.pdf}}

\type{noto-emoji.pdf} (\q{Noto Emoji}) was created from the
\type{svg/} folder of
\goto{\type{googlefonts/noto-emoji@934a5706}}[url(https://github.com/googlefonts/noto-emoji/tree/934a5706)]
and is licensed under the \goto{{\em Apache License}, version
2.0}[url(https://github.com/googlefonts/noto-emoji/blob/934a5706/LICENSE)].

\subsection{\type{twemoji.pdf}}

\type{twemoji.pdf} (\q{Twitter Emoji}) was created from the
\type{assets/svg/} folder of
\goto{\type{twitter/twemoji@d94f4cf7}}[url(https://github.com/twitter/twemoji/tree/d94f4cf7)]
and is licensed under \goto{CC-BY
4.0}[url(https://github.com/twitter/twemoji/blob/d94f4cf7/LICENSE-GRAPHICS)].

\subsection{\type{fxemoji.pdf}}

\type{fxemoji.pdf} (\q{FxEmojis}) was created from the
\type{svgs/FirefoxEmoji/} folder of
\goto{\type{mozilla/fxemoji@270af343}}[url(https://github.com/mozilla/fxemoji/tree/270af343)]
and is licensed under \goto{CC-BY
4.0}[url(https://github.com/mozilla/fxemoji/blob/270af343/LICENSE.md)].

\subsection{\type{openmoji.pdf}}

\type{openmoji.pdf} (\q{OpenMoji}) was created from the
\type{color/svg/} folder of
\goto{\type{hfg-gmuend/openmoji@d6d0daad}}[url(https://github.com/hfg-gmuend/openmoji/tree/d6d0daad)]
and is licensed under \goto{CC-BY-SA
4.0}[url(https://github.com/hfg-gmuend/openmoji/blob/d6d0daad/LICENSE.txt)].

\subsection{\type{emojione.pdf}}

\type{emojione.pdf} (\q{EmojiOne}) was created from the
\type{assets/svg/} folder of
\goto{\type{joypixels/emojione@v2.2.7}}[url(https://github.com/joypixels/emojione/tree/v2.2.7)]
and is licensed under \goto{CC-BY
4.0}[url(https://github.com/joypixels/emojione/blob/v2.2.7/LICENSE.md)].

\page
\setuplayout[
    width=middle,
    backspace=1in,
    height=9.25in,
]
\section[sec:implementation]{Implementation}

From here and until the end of this manual is the raw source code of \unemoji/. This is primarily of interest to developers; most users need not read further.

If want to offer any improvements to the code below, please open an issue or a \acronym{PR} on \goto{GitHub}[url(projecturl)].

\setupbodyfont[10pt]
\setuphead[subsection][
    alternative=normal,
    style=\ssitb,
    after={\blank[disable, penalty:10000]},
    page=yes,
    continue=yes,
]

\subsection{svg-to-pdf.cld}
\typeLUAfile{../source/svg-to-pdf.cld}

\subsection{unnamed-emoji.sty}
\typeTEXfile{../source/unnamed-emoji.sty}

\subsection{unnamed-emoji.tex}
\typeTEXfile{../source/unnamed-emoji.tex}

\subsection{unnamed-emoji.lua}
\typeLUAfile{../source/unnamed-emoji.lua}

\subsection{dvipdfmx.patch}
\typefile{../dvipdfmx.patch}

\stopdocument
