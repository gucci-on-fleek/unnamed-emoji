% unnamed-emoji
% https://github.com/gucci-on-fleek/unnamed-emoji
% SPDX-License-Identifier: MPL-2.0+
% SPDX-FileCopyrightText: 2023 Max Chernoff

% Plain TeX macros for "unnamed-emoji"

\wlog{unnamed-emoji v0.0.4} %%version

% Let us use make "private" macros
\catcode`@=11

% LuaTeX
\ifdefined\directlua
    % Using luatexbase makes LaTeX interop much easier
    \input ltluatex
    % Make /ToUnicode work
    \pdfextension glyphtounicode{fake!}{0000}
    % Load the Lua code
    \directlua{require "unnamed-emoji"}
\fi

% pdfTeX
\ifdefined\pdftexversion
    \def\unemoji@print#1#2{%
        % See if we've already used this emoji
        \ifcsname emoji@#1#2\endcsname%
            % If so, just dereference the XObject
            \pdfrefximage\csname emoji@#1#2\endcsname\relax%
        \else%
            % Otherwise, we first need to load page into an XObject
            \pdfximage named{emoji#2}{#1.pdf}%
            % Save a reference to the XObject
            \expandafter\mathchardef\csname emoji@#1#2\endcsname
                =\pdflastximage\relax%
            % And finally we can print it
            \pdfrefximage\pdflastximage\relax%
        \fi%
    }
\fi

% XeTeX / dvipdfmx
\ifdefined\unemoji@print\else
    \def\unemoji@print#1#2{%
        \hbox to 10bp{% No metrics with dvipdfmx, so just guess 10bp
            % Load and print the emoji, all at once
            \special{pdf:image named "emoji#2" (#1.pdf)}%
            \hss%
        }%
    }%
\fi

% Set the default emoji font
\def\emojifont{noto-emoji}

% Public command
\def\emoji#1{%
    \unemoji@print{\emojifont}{#1}%
}

% Conclusions
\catcode`@=12
\endinput
