% unnamed-emoji
% https://github.com/gucci-on-fleek/unnamed-emoji
% SPDX-License-Identifier: MPL-2.0+
% SPDX-FileCopyrightText: 2023 Max Chernoff

% Plain TeX macros for "unnamed-emoji"

\wlog{unnamed-emoji v0.1.3} %%version

% Let us use make "private" macros
\catcode`@=11

% Knuth "tex" compatibility
\ifx\dimexpr\undefined
    {
        \def\gcsletcs#1#2{%
            \global%
            \expandafter\let%
            \csname#1\expandafter\endcsname%
            \csname#2\endcsname%
        }
        \gcsletcs{ifcsname}{iffalse}
    }
\fi

% Get the required scale factor for the current font
\def\unemoji@getscale{%
    % Can't use \dimexpr since we can't guarantee e-TeX
    \count0=100\relax% Multiplier since we can't use decimals
    \dimen0=1ex\relax% Current font size
    \multiply\count0 by \number\dimen0\relax%
    \divide\count0 by 282168\relax% 1ex in cmr10
    \edef\unemoji@scale|{% ``|'' at the end to prevent gobbling spaces
        \expandafter\unemoji@decimal% Add the decimal point
            \the\count0%
        \end%
    }%
}

% Adds a decimal point to a number
% "100" --> "1.00"; "50" --> "0.50"
\def\unemoji@decimal#1#2#3\end{%
    \ifx\relax#3\relax%
        0.#1#2%
    \else
        #1.#2#3%
    \fi%
}

% The average width of an emoji across all fonts
\newdimen\unemoji@width
\unemoji@width=8.2696pt\relax

% LuaTeX
\ifx\directlua\undefined\else
    % Using luatexbase makes LaTeX interop much easier
    \input ltluatex
    % Make /ToUnicode work
    \pdfextension glyphtounicode{fake!}{0000}
    % Load the Lua code
    \directlua{require "unnamed-emoji"}
\fi

% pdfTeX
\ifx\pdftexversion\undefined\else
    \def\unemoji@print#1#2{%
        \unemoji@getscale% Save the scale factor for the current font
        \setbox0=\hbox{% Place the emoji in a box
            % See if we've already used this emoji
            \ifcsname emoji@#1#2\endcsname%
                % If so, just dereference the XObject
                \pdfrefximage\csname emoji@#1#2\endcsname\relax%
            \else%
                % Otherwise, we first need to load page into an XObject
                \pdfximage named{emoji#2}{unnamed-emoji-#1.pdf}%
                % Save a reference to the XObject
                \global\expandafter\mathchardef\csname emoji@#1#2\endcsname
                    =\pdflastximage\relax%
                % And finally we can print it
                \pdfrefximage\pdflastximage\relax%
            \fi%
        }%
        % Get the emoji's size if it were scaled up
        \dimen0=\unemoji@scale|\dimexpr\wd0\relax%
        % Zero out the box width so that \pdfsave and \pdfrestore are at the
        % same location
        \wd0=0pt\relax%
        % Apply the scaling
        \pdfsave%
            \pdfsetmatrix{\unemoji@scale| 0 0 \unemoji@scale|}%
            \box0%
        \pdfrestore%
        % Now add an empty box with the same width as the scaled emoji
        \hbox to \dimen0{}%
    }
\fi

% XeTeX / dvipdfmx
\ifx\unemoji@print\undefined
    \def\unemoji@print#1#2{%
        \unemoji@getscale%
        \hbox to \unemoji@scale|\unemoji@width{% No metrics with dvipdfmx
            % Load and print the emoji, all at once
            \special{
                pdf:image
                named "emoji#2"
                scale \unemoji@scale|
                (unnamed-emoji-#1.pdf)
            }%
            \hss%
        }%
    }%
\fi

% Set the default emoji font
\def\emojifont{noto-emoji}

% Public command
\def\emoji#1{%
    \leavevmode%
    \unemoji@print{\emojifont}{#1}%
}

% Conclusions
\catcode`@=12
\endinput
